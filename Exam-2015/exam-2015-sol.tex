\documentclass{assignment}
\ProjectInfos{量子光学}{PHYS6251P}{2015 年}{期末考试}{}{陈稼霖}[https://github.com/Chen-Jialin]{SA21038052}

\begin{document}
\begin{prob}
    某一光场的密度算符 $\rho=\sum_n\frac{\langle n\rangle^n}{(1+\langle n\rangle)^{n+1}}\lvert n\rangle\langle n\rvert$, 求其密度算符的 Q 表示.
\end{prob}
\begin{sol}
    该光场的 Q 表示为
    \begin{align}
        \notag Q(\alpha)=&\frac{1}{\pi}\langle\alpha\rvert\rho\lvert\alpha\rangle=\frac{1}{\pi}\langle\alpha\rvert\sum_n\frac{\langle n\rangle}{(1+\langle n\rangle)^{n+1}}\lvert n\rangle\langle n\vert\alpha\rangle\\
        \notag=&\frac{1}{\pi}\sum_n\frac{\langle n\rangle}{(1+\langle n\rangle)^{n+1}}\abs{\langle n\vert\alpha\rangle}^2\\
        \notag=&\frac{1}{\pi}\sum_n\frac{\langle n\rangle}{(1+\langle n\rangle)^{n+1}}\abs{e^{-\abs{\alpha}^2/2}\frac{\alpha^n}{\sqrt{n!}}}^2\\
        \notag=&\frac{1}{\pi}\sum_n\frac{\langle n\rangle}{(1+\langle n\rangle)^{n+1}}e^{-\abs{\alpha}^2}\frac{\abs{\alpha}^{2n}}{n!}\\
        =&\frac{1}{\pi}\frac{\langle n\rangle}{1+\langle n\rangle}\exp\left[-\frac{\langle n\rangle}{1+\langle n\rangle}\abs{\alpha}^2\right].
    \end{align}
\end{sol}

\begin{prob}
    一光场处于这样的态: $\lvert\psi\rangle=Na^{\dagger}\lvert\alpha\rangle$.
    \begin{itemize}
        \item[(1)] 计算归一化常数 $N$.
        \item[(2)] 若 $\alpha$ 为正实数, 判断其取何值时有压缩现象? (提示: 计算 $(\Delta X_1)^2$ 或 $(\Delta X_2)^2$; $X_1=(a+a^{\dagger})/2$, $X_2=(a-a^{\dagger})/2i$).
    \end{itemize}
\end{prob}
\begin{sol}
    \begin{itemize}
        \item[(1)] 由归一化条件,
        \begin{align}
            \notag\langle\psi\vert\psi\rangle=&\abs{N}^2\langle\alpha\rvert aa^{\dagger}\lvert\alpha\rangle\\
            \notag=&\abs{N}^2\langle\alpha\rvert(a^{\dagger}a+1)\lvert\alpha\rangle\\
            \notag=&\abs{N}^2\langle\alpha\rvert(\abs{\alpha}^2+1)\lvert\alpha\rangle\\
            \notag=&\abs{N}^2(\abs{\alpha}^2+1)\\
            =&1,
        \end{align}
        \begin{align}
            \Longrightarrow N=(\abs{\alpha}^2+1)^{-1/2}
        \end{align}
        \item[(2)] 同 2011 年第 4 题.
    \end{itemize}
\end{sol}

\begin{prob}
    某一光场形式为 $\lvert\psi\rangle=\frac{1}{\sqrt{6}}(\lvert 0\rangle+2\lvert 1\rangle+\lvert 2\rangle)$, 判断其是否为亚泊松分布, 为什么?
\end{prob}
\begin{sol}
    该光场的二阶相关度为
    \begin{align}
        \notag g^{(2)}(0)=&\frac{\langle a^{\dagger}a^{\dagger}aa\rangle}{\langle a^{\dagger}a\rangle^2}=\frac{\langle\psi\rvert a^{\dagger}a^{\dagger}aa\lvert\psi\rangle}{\langle\psi\rvert a^{\dagger}a\lvert\psi\rangle^2}\\
        \notag=&6\frac{(\langle 0\rvert+2\langle 1\rvert+\langle 2\rvert)a^{\dagger}a^{\dagger}aa(\lvert 0\rangle+2\lvert 1\rangle+\lvert 2\rangle)}{[(\langle 0\rvert+2\langle 1\rvert+\langle 2\rvert)a^{\dagger}a(\lvert 0\rangle+2\lvert 1\rangle+\lvert 2\rangle)]^2}\\
        \notag=&6\frac{(\langle 0\rvert+2\langle 1\rvert+\langle 2\rvert)(0\lvert 0\rangle+2\cdot 0\lvert 1\rangle+2\lvert 2\rangle)}{[(\langle 0\rvert+2\langle 1\rvert+\langle 2\rvert)(0\lvert 0\rangle+2\cdot 1\lvert 1\rangle+2\lvert 2\rangle)]^2}\\
        =&\frac{1}{3}<1,
    \end{align}
    故该光场为亚泊松分布.
\end{sol}

\begin{prob}
    简述:
    \begin{itemize}
        \item[(1)] 偶极近似的适用条件;
        \item[(2)] 旋转波近似的含义;
        \item[(3)] 马尔科夫近似下的含义;
        \item[(4)] 自发辐射由何引起, 如何抑制或增强;
        \item[(5)] 举例比较光子的一阶干涉和二阶干涉.
    \end{itemize}
\end{prob}
\begin{sol}
    \begin{itemize}
        \item[(1)] 同 2004 年第 3 题 (1).
        \item[(2)] 同 2004 年第 3 题 (2).
        \item[(3)] 同 2004 年第 3 题 (3).
        \item[(4)] 自发辐射由真空中电磁场的涨落引起. 通过添加光学谐振腔或改变光学谐振腔的结构影响光场的模场结构, 进而调控自发辐射的速率.
        \item[(5)] 光子的一阶干涉是光子与其自身的干涉, 体现的是光源的频谱特征 (单色性), 例如迈克耳逊干涉实验.\\
        光子的二阶干涉是光子与光子之间的相干, 体现的是光源的光子数分布特性, 例如 HBT 实验.
    \end{itemize}
\end{sol}

\begin{prob}
    单个二能级原子 (上下能级分别为 $\lvert a\rangle$, $\lvert b\rangle$) 同单模光场 (频率 $\nu=\omega_{ab}$) 共振相互作用. 考虑偶极近似和旋转波近似, 假设相互作用系数为实数.
    \begin{itemize}
        \item[(1)] 写出半经典理论描述的原子-光场系统的总哈密顿量.
        \item[(2)] 写出全量子理论描述的原子-光场系统的总哈密顿量.
        \item[(3)] 原子初态为 $\lvert b\rangle$, 光场初态为 $\lvert 1\rangle$, 利用全量子理论的描述求 $t$ 时刻的原子布局反转数 $W(t)=\abs{c_a}^2-\abs{c_b}^2$.
    \end{itemize}
\end{prob}
\begin{sol}
    同 2004 年第 4 题.
\end{sol}

\begin{prob}
    二能级原子与热平衡辐射场热库相互作用, 其密度算符的运动方程为:
    \begin{align}
        \dot{\rho}=-\frac{\Gamma}{2}[\sigma_+\sigma_-\rho+\rho\sigma_+\sigma_--2\sigma_-\rho\sigma_+].
    \end{align}
    求 $t$ 时刻原子算符 $\langle\sigma_z(t)\rangle$. 提示: $\frac{\mathrm{d}\langle\sigma_z(t)\rangle}{\mathrm{d}t}=\tr[\dot{\rho}\sigma_z]$.
\end{prob}
\begin{sol}
    \begin{align}
        \langle\sigma_z(t)\rangle=\tr[\rho\sigma_z]=\rho_{aa}-\rho_{bb}=2\rho_{aa}-1.
    \end{align}
    一方面,
    \begin{align}
        \Longrightarrow\frac{\mathrm{d}\langle\sigma_z(t)\rangle}{\mathrm{d}t}=2\frac{\mathrm{d}\rho_{aa}}{\mathrm{d}t}-1.
    \end{align}
    另一方面,
    \begin{align}
        \notag\frac{\mathrm{d}\langle\sigma_z(t)\rangle}{\mathrm{d}t}=&\tr[\dot{\rho}\sigma_z]\\
        \notag=&-\frac{\Gamma}{2}\tr[(\sigma_+\sigma_-\rho+\rho\sigma_+\sigma_--2\sigma_-\rho\sigma_+)\sigma_z]\\
        \notag=&-\frac{\Gamma}{2}\tr[\rho(\sigma_z\sigma_+\sigma_-+\sigma_+\sigma_-\sigma_z-2\sigma_+\sigma_z\sigma_-)]\\
        \notag=&-\frac{\Gamma}{2}\tr[\rho(\begin{bmatrix}
            1&0\\
            0&0
        \end{bmatrix}+\begin{bmatrix}
            1&0\\
            0&0
        \end{bmatrix}-2\begin{bmatrix}
            -1&0\\
            0&0
        \end{bmatrix})]\\
        \notag=&-2\Gamma\tr[\rho\begin{bmatrix}
            1&0\\
            0&0
        \end{bmatrix}]\\
        =&-2\Gamma\rho_{aa}.
    \end{align}
    以上两式解联立得
    \begin{align}
        2\frac{\mathrm{d}\rho_{aa}}{\mathrm{d}t}-1=-2\Gamma\rho_{aa},
    \end{align}
    解得
    \begin{align}
        \rho_{aa}=\left(\rho_{aa}(0)-\frac{1}{2\Gamma}\right)e^{-\Gamma t}+\frac{1}{2\Gamma}.
    \end{align}
\end{sol}

\begin{prob}
    简述激光多普勒冷却原子方法的原理.
\end{prob}
\begin{sol}
    
\end{sol}
\end{document}