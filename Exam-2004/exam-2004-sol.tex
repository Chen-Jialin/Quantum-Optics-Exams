\documentclass{assignment}
\ProjectInfos{量子光学}{PHYS6251P}{2004 年}{期末考试}{}{陈稼霖}[https://github.com/Chen-Jialin]{SA21038052}

\begin{document}
\begin{prob}
    单模热光场的密度算符为 $\rho=(1-e^{-\beta})\exp(-\beta a^{\dagger}a)$, $\beta=\hbar\omega/kT$. 求其密度算符的 $Q$ 表示.
\end{prob}
\begin{sol}
    单模热光场的密度算符可化为
    \begin{align}
        \notag\rho=&(1-e^{-\beta})\exp(-\beta a^{\dagger}a)\sum_n\lvert n\rangle\langle n\rvert\\
        \notag=&(1-e^{-\beta})\sum_n\exp(-n\beta)\lvert n\rangle\langle n\rvert.
    \end{align}
    $Q$ 表示为
    \begin{align}
        \notag Q(\alpha)=&\frac{1}{\pi}\langle\alpha\rvert\rho\lvert\alpha\rangle=\langle\alpha\rvert(1-e^{-\beta})\sum_n\exp(-n\beta)\lvert n\rangle\langle n\vert\alpha\rangle\\
        \notag=&\frac{1-e^{-\beta}}{\pi}\sum_n\exp(-n\beta)\abs{\langle n\vert\alpha\rangle}^2\\
        \notag=&\frac{1-e^{-\beta}}{\pi}\sum_n\exp(-n\beta)\abs{e^{-\abs{\alpha}^2/2}\frac{\alpha^n}{\sqrt{n!}}}^2\\
        \notag=&\frac{1-e^{-\beta}}{\pi}e^{-\abs{\alpha}^2}\sum_n\exp(-n\beta)\frac{\abs{\alpha}^{2n}}{n!}\\
        =&\frac{1-e^{-\beta}}{\pi}e^{-\abs{\alpha}^2[1+\exp(-\beta)]}.
    \end{align}
\end{sol}

\begin{prob}
    一光场处于这样的叠加态: $\lvert\psi\rangle=N(\lvert\alpha\rangle+e^{i\theta}\lvert-\alpha\rangle)$.
    \begin{itemize}
        \item[(1)] 计算归一化常数 $N$.
        \item[(2)] 若 $\theta=\pi$, 判断其是否为泊松分布, 为什么?
        \item[(3)] 若 $\theta=0$, $\alpha$ 为纯虚数, 判断其是否有压缩现象, 为什么? (提示: 计算 $(\Delta X_1)^2$, $X_1=(\alpha+\alpha^{\dagger})/2$, $X_2=(\alpha-\alpha^{\dagger})/2i$.)
    \end{itemize}
\end{prob}
\begin{sol}
    \begin{itemize}
        \item[(1)] 由归一化条件,
        \begin{align}
            \notag\langle\psi\vert\psi\rangle=&\abs{N}^2(\langle\alpha\rvert+e^{-i\theta}\langle-\alpha\rvert)(\lvert\alpha\rangle+e^{i\theta}\lvert-\alpha\rangle)\\
            \notag=&\abs{N}^2(\langle\alpha\vert\alpha\rangle+\langle\alpha\vert-\alpha\rangle e^{i\theta}+\langle-\alpha\vert\alpha\rangle e^{-i\theta}+\langle-\alpha\vert-\alpha\rangle)\\
            \notag=&\abs{N}^2\left[1+\exp(-\abs{\alpha}^2)e^{i\theta}+\exp(-\abs{\alpha}^2)e^{-i\theta}+1\right]\\
            \notag=&2\abs{N}^2[1+\exp(-\abs{\alpha}^2)\cos\theta]\\
            =&1,
        \end{align}
        \begin{align}
            \Longrightarrow N=[2+2\exp(-\abs{\alpha}^2)\cos\theta]^{-1/2}.
        \end{align}
        \item[(2)] 若 $\theta=\pi$, 则叠加态为
        \begin{align}
            \lvert\psi\rangle=N(\lvert\alpha\rangle-\lvert-\alpha\rangle),
        \end{align}
        二阶相关度
        \begin{align}
            \notag g^{(2)}(0)=&\frac{\langle a^{\dagger}a^{\dagger}aa\rangle}{\langle a^{\dagger}a\rangle^2}=\frac{\langle\psi\rvert a^{\dagger}a^{\dagger}aa\lvert\psi\rangle}{\langle\psi\rvert a^{\dagger}a\lvert\psi\rangle^2}\\
            \notag=&\frac{\abs{N}^2(\langle\alpha\rvert-\langle-\alpha\rvert)a^{\dagger}a^{\dagger}aa(\lvert\alpha\rangle-\lvert-\alpha\rangle)}{[\abs{N}^2(\langle\alpha\rvert-\langle-\alpha\rvert)a^{\dagger}a(\lvert\alpha\rangle-\lvert-\alpha\rangle)]^2}\\
            \notag=&\frac{[(\alpha^*)^2\langle\alpha\rvert-(\alpha^*)^2\langle-\alpha\rvert][\alpha^2\lvert\alpha\rangle-\alpha^2\lvert-\alpha\rangle]}{\abs{N}^2\{[\alpha^*\langle\alpha\rvert+\alpha^*\langle-\alpha\rvert][\alpha\lvert\alpha\rangle+\alpha\lvert-\alpha\rangle]\}^2}\\
            \notag=&\frac{\abs{\alpha}^4(1-\exp(-\abs{\alpha}^2)-\exp(-\abs{\alpha}^2)+1)}{\abs{N}^2\{\abs{\alpha}^2(1+\exp(-\abs{\alpha}^2)+\exp(-\abs{\alpha}^2)+1)\}^2}\\
            \notag=&\frac{1-\exp(-\abs{\alpha}^2)}{2\abs{N}^2[1+\exp(-\abs{\alpha}^2)]^2}\\
            =&\frac{[1-\exp(-\abs{\alpha}^2)][1+\exp(-\abs{\alpha}^2)\cos\theta]}{[1+\exp(-\abs{\alpha}^2)]^2}<1,
        \end{align}
        故该叠加态为亚泊松分布.
        \item[(3)] 若 $\theta=0$, 叠加态为
        \begin{align}
            \lvert\psi\rangle=N(\lvert\alpha\rangle+\lvert-\alpha\rangle),
        \end{align}
        其中
        \begin{align}
            N=[2+2\exp(-\abs{\alpha}^2)]^{-1/2}
        \end{align}
        $X_1$ 的均值为
        \begin{align}
            \notag\langle X_1\rangle=&N^*(\langle\alpha\rvert+\langle-\alpha\rvert)\frac{1}{2}(a+a^{\dagger})N(\lvert\alpha\rangle+\lvert-\alpha\rangle)\\
            \notag=&\frac{\abs{N}^2}{2}[(\langle\alpha\rvert+\langle-\alpha\rvert)(\alpha\lvert\alpha\rangle-\alpha\lvert-\alpha\rangle)+(\alpha^*\langle\alpha\rvert-\alpha^*\langle-\alpha\rvert)(\lvert\alpha\rangle+\lvert-\alpha\rangle)]\\
            \notag=&\frac{\abs{N}^2}{2}[\alpha(1-\exp(-\abs{\alpha}^2)+\exp(-\abs{\alpha}^2)-1)+\alpha^*(1+\exp(-\abs{\alpha}^2)-\exp(-\abs{\alpha}^2)-1)]\\
            =&0.
        \end{align}
        $X_1^2$ 的均值为
        \begin{align}
            \notag\langle X_1^2\rangle=&N^*(\langle\alpha\rvert+\langle-\alpha\rvert)\left[\frac{1}{2}(a+a^{\dagger})\right]^2N(\lvert\alpha\rangle+\lvert-\alpha\rangle)\\
            \notag=&\frac{\abs{N}^2}{4}(\langle\alpha\rvert+\langle-\alpha\rvert)(aa+aa^{\dagger}+a^{\dagger}a+a^{\dagger}a^{\dagger})(\lvert\alpha\rangle+\lvert-\alpha\rangle)\\
            \notag=&\frac{\abs{N}^2}{4}(\langle\alpha\rvert+\langle-\alpha\rvert)(aa+2a^{\dagger}a+1+a^{\dagger}a^{\dagger})(\lvert\alpha\rangle+\lvert-\alpha\rangle)\\
            \notag=&\frac{\abs{N}^2}{4}[(\langle\alpha\rvert+\langle-\alpha\rvert)(\alpha^2\lvert\alpha\rangle+\alpha^2\lvert-\alpha\rangle)+2(\alpha^*\langle\alpha\rvert-\alpha^*\langle-\alpha\rvert)(\alpha\lvert\alpha\rangle-\alpha\lvert-\alpha\rangle)\\
            \notag&+(\langle\alpha\rvert+\langle-\alpha\rvert)(\lvert\alpha\rangle+\lvert-\alpha\rangle)+((\alpha^*)^2\langle\alpha\rvert+(\alpha^*)^2\langle-\alpha\rvert)(\lvert\alpha\rangle+\lvert-\alpha\rangle)]\\
            \notag=&\frac{\abs{N^2}}{4}[\alpha^2(1+\exp(-\abs{\alpha}^2)+\exp(-\abs{\alpha}^2)+1)+2\abs{\alpha}^2(1-\exp(-\abs{\alpha}^2)-\exp(-\abs{\alpha}^2)+1)\\
            \notag&+(1+\exp(-\abs{\alpha}^2)+\exp(-\abs{\alpha}^2)+1)+(\alpha^*)^2(1+\exp(-\abs{\alpha}^2)+\exp(-\abs{\alpha}^2)+1)]\\
            \notag=&\frac{\abs{N}^2}{2}[(\alpha^2+(\alpha^*)^2+1)(1+\exp(-\abs{\alpha}^2))+\abs{\alpha}^2(1-\exp(-\abs{\alpha}^2)]\\
            =&\frac{1}{4}\left[1+\alpha^2+(\alpha^*)^2+\abs{\alpha}^2\frac{1-\exp(-\abs{\alpha}^2)}{1+\exp(-\abs{\alpha}^2)}\right],
        \end{align}
        由于 $\alpha$ 为纯虚数, 故
        \begin{align}
            \langle X_1^2\rangle=\frac{1}{4}\left[1-\abs{\alpha}^2\frac{1+3\exp(-\abs{\alpha}^2)}{1+\exp(-\abs{\alpha}^2)}\right]\leq\frac{1}{4}.
        \end{align}
        $X_1$ 的涨落为
        \begin{align}
            \Delta X_1=\sqrt{\langle X_1^2\rangle-\langle X_1\rangle^2}=\sqrt{\langle X_1^2\rangle}<\frac{1}{2},
        \end{align}
        故有压缩.
    \end{itemize}
\end{sol}

\begin{prob}
    简述:
    \begin{itemize}
        \item[(1)] 偶极近似的适用条件;
        \item[(2)] 旋转波近似的含义;
        \item[(3)] 马尔可夫近似的含义;
        \item[(4)] 自发辐射和受激辐射的不同点;
        \item[(5)] Hanbry-Brown-Twiss 实验与迈克尔逊干涉实验的不同之处.
    \end{itemize}
\end{prob}
\begin{sol}
    \begin{itemize}
        \item[(1)] 偶极近似的适用条件: 讨论的系统尺寸远小于光的波长.
        \item[(2)] 旋转波近似的含义: 光场频率相对于系统的本征频率具有大失谐的分量在足够长时间的积分下对系统演化的贡献平均为零.
        \item[(3)] 马尔科夫近似: 系统无记忆, 系统的当前状态与其历史无关, 即关联时间为 $0$, 该近似适用于热库非常大且系统与热库相互作用持续的情形.
        \item[(4)] 
        \begin{itemize}
            \item[(a)] 自发辐射无需外界辐射场的激励就可发生, 受激辐射需要外界辐射场的激励才能发生;
            \item[(b)] 自发辐射的发生时间、相位、方向和偏振是随机的, 受激辐射的相位、方向和偏振与诱导光子相同.
        \end{itemize}
        \item[(5)] HBT 实验是二阶相干, 是光子与光子之间的相干, 体现了光源的光子数分布特性; 迈克尔逊干涉实验是一阶相干, 是光子自身的相干, 体现了光源频谱的单色性.
    \end{itemize}
\end{sol}

\begin{prob}
    单个二能级原子 (上下能级分别为 $\lvert a\rangle$, $\lvert b\rangle$) 同单模光场 (频率 $\nu=\omega_{ab}$) 共振相互作用. 考虑偶极近似和旋转波近似, 假设相互作用系数为实数.
    \begin{itemize}
        \item[(1)] 写出半经典理论描述的原子-光场系统哈密顿量.
        \item[(2)] 写出全量子理论描述的原子-光场哈密顿量.
        \item[(3)] 原子初态为 $\lvert a\rangle$, 光场初态为真空态, 利用全量子理论的描述求 $t$ 时刻的原子布局数反转数 $W(t)=\abs{c_a}^2-\abs{c_b}^2$.
        \item[(4)] (附加题) 若原子初态为 $\frac{1}{\sqrt{2}}(\lvert a\rangle-\lvert b\rangle)$, 光场的初态为真空态, 求 $W(t)$.
    \end{itemize}
\end{prob}
\begin{sol}
    \begin{itemize}
        \item[(1)] 
        \item[(2)] 
        \item[(3)] 
        \item[(4)] 
    \end{itemize}
\end{sol}

\begin{prob}
    单模光场与热平衡辐射场热库相互作用, 其密度算符的运动方程为:
    \[
        \rho=-\frac{\gamma}{2}\bar{n}(aa^{\dagger}\rho-2a^{\dagger}\rho a+\rho aa^{\dagger})-\frac{\gamma}{2}(\bar{n}+1)(a^{\dagger}a\rho-2a\rho a^{\dagger}+\rho a^{\dagger}a).
    \]
    求 $t$ 时刻光场的粒子数平均值 $\langle a^{\dagger}(t)a(t)\rangle$. [提示: $\frac{\mathrm{d}\langle a^{\dagger}a\rangle}{\mathrm{d}t}=\tr(\rho a^{\dagger}a)$]
\end{prob}
\begin{sol}
    
\end{sol}

\begin{prob}
    单模光场与热平衡辐射场热库相互作用的 Langevin 方程为
    \[
        \tilde{a}(t)=-\frac{\gamma}{2}\tilde{a}(t)+F_a(t),
    \]
    且已知: $\langle F_{\tilde{a}}^{\dagger}(t)\tilde{a}(t)\rangle_R+\langle\tilde{a}^{\dagger}(t)F_{\tilde{a}}(t)\rangle=\gamma\bar{n}$. 试求 $\langle\tilde{a}^{\dagger}(t)\tilde{a}(t)\rangle_R$.
\end{prob}
\begin{sol}
    
\end{sol}

\begin{prob}[附加题]
    如何理解原子谱线宽度和寿命的不确定性关系 $\Delta\omega\propto 1/\tau$.
\end{prob}
\begin{sol}
    原子上能级具有有限的寿命 $\tau$, 由不确定性原理, 其能量值具有不确定性 $\Delta E$, 满足
    \begin{align}
        \Delta E\cdot\tau\sim\frac{\hbar}{2},
    \end{align}
    故原子谱线, 即原子由上能级跃迁至下能级辐射的光子频率具有不确定性
    \begin{align}
        \Delta\omega=\frac{\Delta E}{\hbar}\propto\frac{1}{\tau}.
    \end{align}
\end{sol}

\begin{prob}[附加题]
    谈谈你对信息的理解.
\end{prob}
\begin{sol}
    信息即不确定度的减少, 信息的数量用香农熵
    \begin{align}
        H=-\sum_{k=1}^np_k\log_2p_k
    \end{align}
    来衡量.
\end{sol}

\begin{prob}[附加题]
    简述量子光学主要是研究什么问题? 在经典物理学中主要用哪一些学科研究这些内容 (例如可举普通物理和四大力学中的学科和工具).
\end{prob}
\begin{sol}
    量子光学主要研究光场的量子化和光与物质的相互作用.

    经典物理中有光学、原子物理学、电动力学、量子力学来研究这些内容.
\end{sol}
\end{document}