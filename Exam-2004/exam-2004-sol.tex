\documentclass{assignment}
\ProjectInfos{量子光学}{PHYS6251P}{2004 年}{期末考试}{}{陈稼霖}[https://github.com/Chen-Jialin]{SA21038052}

\begin{document}
\begin{prob}
    单模热光场的密度算符为 $\rho=(1-e^{-\beta})\exp(-\beta a^{\dagger}a)$, $\beta=\hbar\omega/kT$. 求其密度算符的 $Q$ 表示.
\end{prob}
\begin{sol}
    单模热光场的密度算符可化为
    \begin{align}
        \notag\rho=&(1-e^{-\beta})\exp(-\beta a^{\dagger}a)\sum_n\lvert n\rangle\langle n\rvert\\
        \notag=&(1-e^{-\beta})\sum_n\exp(-n\beta)\lvert n\rangle\langle n\rvert.
    \end{align}
    $Q$ 表示为
    \begin{align}
        \notag Q(\alpha)=&\frac{1}{\pi}\langle\alpha\rvert\rho\lvert\alpha\rangle=\frac{1}{\pi}\langle\alpha\rvert(1-e^{-\beta})\sum_n\exp(-n\beta)\lvert n\rangle\langle n\vert\alpha\rangle\\
        \notag=&\frac{1-e^{-\beta}}{\pi}\sum_n\exp(-n\beta)\abs{\langle n\vert\alpha\rangle}^2\\
        \notag=&\frac{1-e^{-\beta}}{\pi}\sum_n\exp(-n\beta)\abs{e^{-\abs{\alpha}^2/2}\frac{\alpha^n}{\sqrt{n!}}}^2\\
        \notag=&\frac{1-e^{-\beta}}{\pi}e^{-\abs{\alpha}^2}\sum_n\exp(-n\beta)\frac{\abs{\alpha}^{2n}}{n!}\\
        =&\frac{1-e^{-\beta}}{\pi}e^{-\abs{\alpha}^2[1+\exp(-\beta)]}.
    \end{align}
\end{sol}

\begin{prob}
    一光场处于这样的叠加态: $\lvert\psi\rangle=N(\lvert\alpha\rangle+e^{i\theta}\lvert-\alpha\rangle)$.
    \begin{itemize}
        \item[(1)] 计算归一化常数 $N$.
        \item[(2)] 若 $\theta=\pi$, 判断其是否为泊松分布, 为什么?
        \item[(3)] 若 $\theta=0$, $\alpha$ 为纯虚数, 判断其是否有压缩现象, 为什么? (提示: 计算 $(\Delta X_1)^2$, $X_1=(\alpha+\alpha^{\dagger})/2$, $X_2=(\alpha-\alpha^{\dagger})/2i$.)
    \end{itemize}
\end{prob}
\begin{sol}
    \begin{itemize}
        \item[(1)] 由归一化条件,
        \begin{align}
            \notag\langle\psi\vert\psi\rangle=&\abs{N}^2(\langle\alpha\rvert+e^{-i\theta}\langle-\alpha\rvert)(\lvert\alpha\rangle+e^{i\theta}\lvert-\alpha\rangle)\\
            \notag=&\abs{N}^2(\langle\alpha\vert\alpha\rangle+\langle\alpha\vert-\alpha\rangle e^{i\theta}+\langle-\alpha\vert\alpha\rangle e^{-i\theta}+\langle-\alpha\vert-\alpha\rangle)\\
            \notag=&\abs{N}^2\left[1+\exp(-\abs{\alpha}^2)e^{i\theta}+\exp(-\abs{\alpha}^2)e^{-i\theta}+1\right]\\
            \notag=&2\abs{N}^2[1+\exp(-\abs{\alpha}^2)\cos\theta]\\
            =&1,
        \end{align}
        \begin{align}
            \Longrightarrow N=[2+2\exp(-\abs{\alpha}^2)\cos\theta]^{-1/2}.
        \end{align}
        \item[(2)] 若 $\theta=\pi$, 则叠加态为
        \begin{align}
            \lvert\psi\rangle=N(\lvert\alpha\rangle-\lvert-\alpha\rangle),
        \end{align}
        二阶相关度
        \begin{align}
            \notag g^{(2)}(0)=&\frac{\langle a^{\dagger}a^{\dagger}aa\rangle}{\langle a^{\dagger}a\rangle^2}=\frac{\langle\psi\rvert a^{\dagger}a^{\dagger}aa\lvert\psi\rangle}{\langle\psi\rvert a^{\dagger}a\lvert\psi\rangle^2}\\
            \notag=&\frac{\abs{N}^2(\langle\alpha\rvert-\langle-\alpha\rvert)a^{\dagger}a^{\dagger}aa(\lvert\alpha\rangle-\lvert-\alpha\rangle)}{[\abs{N}^2(\langle\alpha\rvert-\langle-\alpha\rvert)a^{\dagger}a(\lvert\alpha\rangle-\lvert-\alpha\rangle)]^2}\\
            \notag=&\frac{[(\alpha^*)^2\langle\alpha\rvert-(\alpha^*)^2\langle-\alpha\rvert][\alpha^2\lvert\alpha\rangle-\alpha^2\lvert-\alpha\rangle]}{\abs{N}^2\{[\alpha^*\langle\alpha\rvert+\alpha^*\langle-\alpha\rvert][\alpha\lvert\alpha\rangle+\alpha\lvert-\alpha\rangle]\}^2}\\
            \notag=&\frac{\abs{\alpha}^4(1-\exp(-\abs{\alpha}^2)-\exp(-\abs{\alpha}^2)+1)}{\abs{N}^2\{\abs{\alpha}^2(1+\exp(-\abs{\alpha}^2)+\exp(-\abs{\alpha}^2)+1)\}^2}\\
            \notag=&\frac{1-\exp(-\abs{\alpha}^2)}{2\abs{N}^2[1+\exp(-\abs{\alpha}^2)]^2}\\
            =&\frac{[1-\exp(-\abs{\alpha}^2)][1+\exp(-\abs{\alpha}^2)\cos\theta]}{[1+\exp(-\abs{\alpha}^2)]^2}<1,
        \end{align}
        故该叠加态为亚泊松分布.
        \item[(3)] 若 $\theta=0$, 叠加态为
        \begin{align}
            \lvert\psi\rangle=N(\lvert\alpha\rangle+\lvert-\alpha\rangle),
        \end{align}
        其中
        \begin{align}
            N=[2+2\exp(-\abs{\alpha}^2)]^{-1/2}
        \end{align}
        $X_1$ 的均值为
        \begin{align}
            \notag\langle X_1\rangle=&N^*(\langle\alpha\rvert+\langle-\alpha\rvert)\frac{1}{2}(a+a^{\dagger})N(\lvert\alpha\rangle+\lvert-\alpha\rangle)\\
            \notag=&\frac{\abs{N}^2}{2}[(\langle\alpha\rvert+\langle-\alpha\rvert)(\alpha\lvert\alpha\rangle-\alpha\lvert-\alpha\rangle)+(\alpha^*\langle\alpha\rvert-\alpha^*\langle-\alpha\rvert)(\lvert\alpha\rangle+\lvert-\alpha\rangle)]\\
            \notag=&\frac{\abs{N}^2}{2}[\alpha(1-\exp(-\abs{\alpha}^2)+\exp(-\abs{\alpha}^2)-1)+\alpha^*(1+\exp(-\abs{\alpha}^2)-\exp(-\abs{\alpha}^2)-1)]\\
            =&0.
        \end{align}
        $X_1^2$ 的均值为
        \begin{align}
            \notag\langle X_1^2\rangle=&N^*(\langle\alpha\rvert+\langle-\alpha\rvert)\left[\frac{1}{2}(a+a^{\dagger})\right]^2N(\lvert\alpha\rangle+\lvert-\alpha\rangle)\\
            \notag=&\frac{\abs{N}^2}{4}(\langle\alpha\rvert+\langle-\alpha\rvert)(aa+aa^{\dagger}+a^{\dagger}a+a^{\dagger}a^{\dagger})(\lvert\alpha\rangle+\lvert-\alpha\rangle)\\
            \notag=&\frac{\abs{N}^2}{4}(\langle\alpha\rvert+\langle-\alpha\rvert)(aa+2a^{\dagger}a+1+a^{\dagger}a^{\dagger})(\lvert\alpha\rangle+\lvert-\alpha\rangle)\\
            \notag=&\frac{\abs{N}^2}{4}[(\langle\alpha\rvert+\langle-\alpha\rvert)(\alpha^2\lvert\alpha\rangle+\alpha^2\lvert-\alpha\rangle)+2(\alpha^*\langle\alpha\rvert-\alpha^*\langle-\alpha\rvert)(\alpha\lvert\alpha\rangle-\alpha\lvert-\alpha\rangle)\\
            \notag&+(\langle\alpha\rvert+\langle-\alpha\rvert)(\lvert\alpha\rangle+\lvert-\alpha\rangle)+((\alpha^*)^2\langle\alpha\rvert+(\alpha^*)^2\langle-\alpha\rvert)(\lvert\alpha\rangle+\lvert-\alpha\rangle)]\\
            \notag=&\frac{\abs{N^2}}{4}[\alpha^2(1+\exp(-\abs{\alpha}^2)+\exp(-\abs{\alpha}^2)+1)+2\abs{\alpha}^2(1-\exp(-\abs{\alpha}^2)-\exp(-\abs{\alpha}^2)+1)\\
            \notag&+(1+\exp(-\abs{\alpha}^2)+\exp(-\abs{\alpha}^2)+1)+(\alpha^*)^2(1+\exp(-\abs{\alpha}^2)+\exp(-\abs{\alpha}^2)+1)]\\
            \notag=&\frac{\abs{N}^2}{2}[(\alpha^2+(\alpha^*)^2+1)(1+\exp(-\abs{\alpha}^2))+\abs{\alpha}^2(1-\exp(-\abs{\alpha}^2)]\\
            =&\frac{1}{4}\left[1+\alpha^2+(\alpha^*)^2+\abs{\alpha}^2\frac{1-\exp(-\abs{\alpha}^2)}{1+\exp(-\abs{\alpha}^2)}\right],
        \end{align}
        由于 $\alpha$ 为纯虚数, 故
        \begin{align}
            \langle X_1^2\rangle=\frac{1}{4}\left[1-\abs{\alpha}^2\frac{1+3\exp(-\abs{\alpha}^2)}{1+\exp(-\abs{\alpha}^2)}\right]\leq\frac{1}{4}.
        \end{align}
        $X_1$ 的涨落为
        \begin{align}
            \Delta X_1=\sqrt{\langle X_1^2\rangle-\langle X_1\rangle^2}=\sqrt{\langle X_1^2\rangle}<\frac{1}{2},
        \end{align}
        故有压缩.
    \end{itemize}
\end{sol}

\begin{prob}
    简述:
    \begin{itemize}
        \item[(1)] 偶极近似的适用条件;
        \item[(2)] 旋转波近似的含义;
        \item[(3)] 马尔可夫近似的含义;
        \item[(4)] 自发辐射和受激辐射的不同点;
        \item[(5)] Hanbry-Brown-Twiss 实验与迈克尔逊干涉实验的不同之处.
    \end{itemize}
\end{prob}
\begin{sol}
    \begin{itemize}
        \item[(1)] 偶极近似的适用条件: 讨论的系统尺寸远小于光的波长.
        \item[(2)] 旋转波近似的含义: 光场频率相对于系统的本征频率具有大失谐的分量在足够长时间的积分下对系统演化的贡献平均为零.
        \item[(3)] 马尔科夫近似: 系统无记忆, 系统的当前状态与其历史无关, 即关联时间为 $0$, 该近似适用于热库非常大且系统与热库相互作用持续的情形.
        \item[(4)] 
        \begin{itemize}
            \item[(a)] 自发辐射无需外界辐射场的激励就可发生, 受激辐射需要外界辐射场的激励才能发生;
            \item[(b)] 自发辐射的发生时间、相位、方向和偏振是随机的, 受激辐射的相位、方向和偏振与诱导光子相同.
        \end{itemize}
        \item[(5)] HBT 实验是二阶相干, 是光子与光子之间的相干, 体现了光源的光子数分布特性; 迈克尔逊干涉实验是一阶相干, 是光子自身的相干, 体现了光源频谱的单色性.
    \end{itemize}
\end{sol}

\begin{prob}
    单个二能级原子 (上下能级分别为 $\lvert a\rangle$, $\lvert b\rangle$) 同单模光场 (频率 $\nu=\omega_{ab}$) 共振相互作用. 考虑偶极近似和旋转波近似, 假设相互作用系数为实数.
    \begin{itemize}
        \item[(1)] 写出半经典理论描述的原子-光场系统哈密顿量.
        \item[(2)] 写出全量子理论描述的原子-光场哈密顿量.
        \item[(3)] 原子初态为 $\lvert a\rangle$, 光场初态为真空态, 利用全量子理论的描述求 $t$ 时刻的原子布居数反转数 $W(t)=\abs{c_a}^2-\abs{c_b}^2$.
        \item[(4)] (附加题) 若原子初态为 $\frac{1}{\sqrt{2}}(\lvert a\rangle-\lvert b\rangle)$, 光场的初态为真空态, 求 $W(t)$.
    \end{itemize}
\end{prob}
\begin{sol}
    \begin{itemize}
        \item[(1)] 半经典理论描述的原子-光场系统哈密顿量为
        \begin{align}
            H=\hbar\omega_a\lvert a\rangle\langle a\rvert+\hbar\omega_b\lvert b\rangle\langle b\rvert-(\mu_{ab}\lvert a\rangle\langle b\rvert+\mu_{ba}\lvert b\rangle\langle a\rvert)\mathcal{E}\cos\nu t
        \end{align}
        其中电偶极矩算符的对角项
        \begin{align}
            \mu_{ab}=\langle a\rvert\mu\lvert b\rangle,\quad\mu_{ba}=\langle b\rvert\mu\lvert a\rangle.
        \end{align}
        \item[(2)] 全量子理论描述的原子-光场哈密顿量为
        \begin{align}
            H=H_0+H_1.
        \end{align}
        其中无微扰哈密顿量
        \begin{align}
            H_0=\hbar\nu a^{\dagger}a+\frac{1}{2}\hbar\omega_{ab}\sigma_z,
        \end{align}
        相互作用哈密顿量
        \begin{align}
            H_1=\hbar g(\sigma_+a+a^{\dagger}\sigma_-).
        \end{align}
        \item[(3)] 相互作用表象下, 系统的相互作用哈密顿量为
        \begin{align}
            \hat{V}=e^{i\hat{H}_0t/\hbar}H_1e^{-i\hat{H}_0t/\hbar}=\hbar g(\sigma_+ae^{i\Delta t}+a^{\dagger}\sigma_-e^{-i\Delta t}),
        \end{align}
        其中频率失谐 $\Delta=\omega_{ab}-\nu$.
        系统演化遵循相互作用表象下薛定谔方程:
        \begin{align}
            i\hbar\frac{\partial}{\partial t}\lvert\psi(t)\rangle=\hat{V}\lvert\psi(t)\rangle,
        \end{align}
        代入系统的量子态
        \begin{align}
            \lvert\psi(t)\rangle=\sum_{n=0}^{\infty}[C_{a,n}(t)\lvert a,n\rangle+C_{b,n}\lvert b,n\rangle],
        \end{align}
        有
        \begin{align}
            \dot{C}_{a,n}(t)=&-igC_{b,n+1}\sqrt{n+1}e^{i\Delta t},\\
            \dot{C}_{b,n+1}(t)=&-igC_{a,n}\sqrt{n+1}e^{-i\Delta t},
        \end{align}
        解得
        \begin{align}
            C_{a,n}(t)=&\left\{C_{a,n}(0)\left[\cos\left(\frac{\Omega_nt}{2}\right)-\frac{i\Delta}{\Omega_n}\sin\left(\frac{\Omega_nt}{2}\right)\right]-\frac{2ig\sqrt{n+1}}{\Omega_n}C_{b,n+1}(0)\sin\left(\frac{\Omega_nt}{2}\right)\right\}e^{i\Delta t/2},\\
            C_{b,n+1}(t)=&\left\{C_{b,n+1}(0)\left[\cos\left(\frac{\Omega_nt}{2}\right)+\frac{i\Delta}{\Omega_n}\sin\left(\frac{\Omega_nt}{2}\right)\right]-\frac{2ig\sqrt{n+1}}{\Omega_n}C_{a,n}(0)\sin\left(\frac{\Omega_nt}{2}\right)\right\}e^{-i\Delta t/2},
        \end{align}
        其中
        \begin{align}
            \Omega_n^2=\Delta^2+4g^2(n+1).
        \end{align}
        原子初态为 $\lvert a\rangle$, 即 $C_{a,0}=1$, 将其代入上面两式得
        \begin{align}
            C_{a,0}(t)=&\left[\cos\left(\frac{\Omega_0t}{2}\right)-\frac{i\Delta}{\Omega_0}\sin\left(\frac{\Omega_0t}{2}\right)\right]e^{i\Delta t/2},\\
            C_{b,1}(t)=&-\frac{2ig}{\Omega_0}\sin\left(\frac{\Omega_0t}{2}\right)e^{-i\Delta t/2}.
        \end{align}
        $t$ 时刻的原子布居数反转为
        \begin{align}
            W(t)=\abs{c_a}^2-\abs{c_b}^2=\abs{C_{a,0}(t)}^2-\abs{C_{b,1}(t)}^2=\cos^2\left(\frac{\Omega_0t}{2}\right)+\frac{\Delta^2-4g^2}{\Omega_0^2}\sin^2\left(\frac{\Omega_0t}{2}\right).
        \end{align}
        由于 $\nu=\omega_{ab}$, 故 $\Delta=0$, $\Omega_0^2=4g^2$,
        \begin{align}
            W(t)=\cos^2(gt)-\sin^2(gt)=\cos(2gt).
        \end{align}
        \item[(4)] 若原子初态为 $\frac{1}{\sqrt{2}}(\lvert a\rangle-\lvert b\rangle)$, 光场的初态为真空态, 则 $C_{a,0}(0)=\frac{1}{\sqrt{2}}$, $C_{b,0}=\frac{1}{\sqrt{2}}$,
        \begin{align}
            C_{a,0}(t)=&\frac{1}{\sqrt{2}}\left[\cos\left(\frac{\Omega_0t}{2}\right)-\frac{i\Delta}{\Omega_0}\sin\left(\frac{\Omega_0t}{2}\right)\right]e^{i\Delta t/2},\\
            C_{b,1}(t)=&-\frac{\sqrt{2}ig}{\Omega_0}\sin\left(\frac{\Omega_0t}{2}\right)e^{-i\Delta t/2},\\
            C_{b,0}(t)=&\frac{1}{\sqrt{2}}.
        \end{align}
        $t$ 时刻的原子布居数反转为
        \begin{align}
            \notag W(t)=&\abs{c_a(t)}^2-\abs{c_b(t)}^2=\abs{C_{a,0}(t)}^2-\abs{C_{b,1}}^2-\abs{C_{b,0}(t)}^2\\
            =&\frac{1}{2}\cos^2\left(\frac{\Omega_0t}{2}\right)+\frac{\Delta^2-4g^2}{2\Omega_0^2}\sin^2\left(\frac{\Omega_0t}{2}\right)-\frac{1}{2}.
        \end{align}
        由于 $\nu=\omega_{ab}$, 故 $\Delta=0$, $\Omega_n^2=4g^2(n+1)$,
        \begin{align}
            W(t)=\frac{1}{2}\cos^2(gt)-\frac{1}{2}\sin^2(gt)-\frac{1}{2}=\frac{\cos(2gt)-1}{2}.
        \end{align}
    \end{itemize}
\end{sol}

\begin{prob}
    单模光场与热平衡辐射场热库相互作用, 其密度算符的运动方程为:
    \[
        \rho=-\frac{\gamma}{2}\bar{n}(aa^{\dagger}\rho-2a^{\dagger}\rho a+\rho aa^{\dagger})-\frac{\gamma}{2}(\bar{n}+1)(a^{\dagger}a\rho-2a\rho a^{\dagger}+\rho a^{\dagger}a).
    \]
    求 $t$ 时刻光场的粒子数平均值 $\langle a^{\dagger}(t)a(t)\rangle$. [提示: $\frac{\mathrm{d}\langle a^{\dagger}a\rangle}{\mathrm{d}t}=\tr(\rho a^{\dagger}a)$]
\end{prob}
\begin{sol}
    \begin{align}
        \notag&\frac{\mathrm{d}\langle a^{\dagger}a\rangle}{\mathrm{d}t}=\tr(\rho a^{\dagger}a)=\tr(\dot{\rho}a^{\dagger}a)\\
        \notag=&\tr\left\{\left[-\frac{\gamma}{2}\bar{n}(aa^{\dagger}\rho-2a^{\dagger}\rho a+\rho aa^{\dagger})-\frac{\gamma}{2}(\bar{n}+1)(a^{\dagger}a\rho-2a\rho a^{\dagger}+\rho a^{\dagger}a)\right]a^{\dagger}a\right\}\\
        \notag=&\tr\left\{-\frac{\gamma}{2}\bar{n}(aa^{\dagger}\rho a^{\dagger}a-2a^{\dagger}\rho aa^{\dagger}a+\rho aa^{\dagger}a^{\dagger}a)-\frac{\gamma}{2}(\bar{n}+1)(a^{\dagger}a\rho a^{\dagger}a-2a\rho a^{\dagger}a^{\dagger}a+\rho a^{\dagger}aa^{\dagger}a)\right\}\\
        \notag=&\tr\left\{\rho\left[-\frac{\gamma}{2}\bar{n}(a^{\dagger}aaa^{\dagger}-2aa^{\dagger}aa^{\dagger}+aa^{\dagger}a^{\dagger}a)-\frac{\gamma}{2}(\bar{n}+1)(a^{\dagger}aa^{\dagger}a-2a^{\dagger}a^{\dagger}aa+a^{\dagger}aa^{\dagger}a)\right]\right\}\\
        \notag=&\tr\left\{\rho\left[-\frac{\gamma}{2}\bar{n}(a^{\dagger}a(a^{\dagger}a+1)-2(a^{\dagger}a+1)(a^{\dagger}a+1)+(a^{\dagger}a+1)a^{\dagger}a)-\frac{\gamma}{2}(\bar{n}+1)(2a^{\dagger}(a^{\dagger}a+1)a-2a^{\dagger}a^{\dagger}aa)\right]\right\}\\
        \notag=&\tr\left\{\rho\left[\gamma\bar{n}(a^{\dagger}a+1)-\gamma(\bar{n}+1)a^{\dagger}a\right]\right\}\\
        \notag=&\tr\left\{\gamma\rho(\bar{n}-a^{\dagger}a)\right\}\\
        =&\gamma(\bar{n}-\langle a^{\dagger}a\rangle),
    \end{align}
    解得 $t$ 时刻光场的粒子数平均值为
    \begin{align}
        \langle a^{\dagger}(t)a(t)\rangle=(\langle a^{\dagger}(0)a(0)\rangle-\bar{n})e^{-\gamma t}+\bar{n}.
    \end{align}
\end{sol}

\begin{prob}
    单模光场与热平衡辐射场热库相互作用的 Langevin 方程为
    \[
        \dot{\tilde{a}}(t)=-\frac{\gamma}{2}\tilde{a}(t)+F_{\tilde{a}}(t),
    \]
    且已知: $\langle F_{\tilde{a}}^{\dagger}(t)\tilde{a}(t)\rangle_R+\langle\tilde{a}^{\dagger}(t)F_{\tilde{a}}(t)\rangle_R=\gamma\bar{n}$. 试求 $\langle\tilde{a}^{\dagger}(t)\tilde{a}(t)\rangle_R$.
\end{prob}
\begin{sol}
    Langevin 方程的共轭为
    \begin{align}
        \dot{\tilde{a}}^{\dagger}(t)=-\frac{\gamma^*}{2}\tilde{a}^{\dagger}(t)+F_{\tilde{a}}^{\dagger}(t).
    \end{align}
    $\langle\tilde{a}^{\dagger}(t)\tilde{a}(t)\rangle$ 遵循演化方程:
    \begin{align}
        \notag\frac{\mathrm{d}\langle\tilde{a}^{\dagger}(t)\tilde{a}(t)\rangle_R}{\mathrm{d}t}=&\left\langle\frac{\mathrm{d}\tilde{a}^{\dagger}(t)\tilde{a}(t)}{\mathrm{d}t}\right\rangle_R=\left\langle\frac{\mathrm{d}\tilde{a}^{\dagger}(t)}{\mathrm{d}t}\tilde{a}(t)+\tilde{a}^{\dagger}(t)\frac{\mathrm{d}\tilde{a}(t)}{\mathrm{d}t}\right\rangle_R\\
        \notag=&\left\langle\left[-\frac{\gamma^*}{2}\tilde{a}^{\dagger}(t)+F_{\tilde{a}}^{\dagger}(t)\right]\tilde{a}(t)+\tilde{a}^{\dagger}(t)\left[-\frac{\gamma}{2}\tilde{a}(t)+F_{\tilde{a}}(t)\right]\right\rangle_R\\
        \notag=&-\frac{\gamma^*+\gamma}{2}\langle\tilde{a}^{\dagger}(t)\tilde{a}(t)\rangle_R+\langle F_{\tilde{a}}^{\dagger}(t)\tilde{a}(t)\rangle_R+\langle\tilde{a}^{\dagger}(t)F_{\tilde{a}}(t)\rangle_R\\
        =&-\frac{\gamma^*+\gamma}{2}\langle\tilde{a}^{\dagger}(t)\tilde{a}(t)\rangle_R+\gamma\bar{n},
    \end{align}
    解得
    \begin{align}
        \langle\tilde{a}^{\dagger}(t)\tilde{a}(t)\rangle_R=\left[\langle\tilde{a}^{\dagger}(0)\tilde{a}(0)\rangle-\frac{2\gamma}{\gamma^*+\gamma}\right]e^{-\frac{\gamma^*+\gamma}{2}t}+\frac{2\gamma}{\gamma^*+\gamma}\bar{n}.
    \end{align}
\end{sol}

\begin{prob}[附加题]
    如何理解原子谱线宽度和寿命的不确定性关系 $\Delta\omega\propto 1/\tau$.
\end{prob}
\begin{sol}
    原子上能级具有有限的寿命 $\tau$, 由不确定性原理, 其能量值具有不确定性 $\Delta E$, 满足
    \begin{align}
        \Delta E\cdot\tau\sim\frac{\hbar}{2},
    \end{align}
    故原子谱线, 即原子由上能级跃迁至下能级辐射的光子频率具有不确定性
    \begin{align}
        \Delta\omega=\frac{\Delta E}{\hbar}\propto\frac{1}{\tau}.
    \end{align}
\end{sol}

\begin{prob}[附加题]
    谈谈你对信息的理解.
\end{prob}
\begin{sol}
    信息即不确定度的减少, 信息的量用香农熵
    \begin{align}
        H=-\sum_{k=1}^np_k\log_2p_k
    \end{align}
    来衡量.
\end{sol}

\begin{prob}[附加题]
    简述量子光学主要是研究什么问题? 在经典物理学中主要用哪一些学科研究这些内容 (例如可举普通物理和四大力学中的学科和工具).
\end{prob}
\begin{sol}
    量子光学主要研究光场的量子化和光与物质的相互作用.

    经典物理学中有光学、原子物理学、电动力学来研究这些内容.
\end{sol}
\end{document}