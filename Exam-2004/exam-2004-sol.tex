\documentclass{assignment}
\ProjectInfos{量子光学}{PHYS6251P}{2004 年}{期末考试}{}{陈稼霖}[https://github.com/Chen-Jialin]{SA21038052}

\begin{document}
\begin{prob}
    单模热光场的密度算符为 $\rho=(1-e^{-\beta})\exp(-\beta a^{\dagger}a)$, $\beta=\hbar\omega/kT$. 求其密度算符的 $Q$ 表示.
\end{prob}
\begin{sol}

\end{sol}

\begin{prob}
    一光场处于这样的叠加态: $\lvert\psi\rangle=N(\lvert\alpha\rangle+e^{i\theta}\lvert-\alpha\rangle)$.
    \begin{itemize}
        \item[(1)] 计算归一化常数 $N$.
        \item[(2)] 若 $\theta=\pi$, 判断其是否为泊松分布, 为什么?
        \item[(3)] 若 $\theta=0$, $\alpha$ 为纯虚数, 判断其是否有压缩现象, 为什么? (提示: 计算 $(\Delta X_1)^2$, $X_1=(\alpha+\alpha^{\dagger})/2$, $X_2=(\alpha-\alpha^{\dagger})/2i$.)
    \end{itemize}
\end{prob}
\begin{sol}
    \begin{itemize}
        \item[(1)] 
        \item[(2)] 
        \item[(3)] 
    \end{itemize}
\end{sol}

\begin{prob}
    简述:
    \begin{itemize}
        \item[(1)] 偶极近似的适用条件;
        \item[(2)] 旋转波近似的含义;
        \item[(3)] 马尔可夫近似的含义;
        \item[(4)] 自发辐射和受激辐射的不同点;
        \item[(5)] Hanbry-Brown-Twiss 实验与迈克尔逊干涉实验的不同之处.
    \end{itemize}
\end{prob}
\begin{sol}
    \begin{itemize}
        \item[(1)] 
        \item[(2)] 
        \item[(3)] 
        \item[(4)] 
        \item[(5)] 
    \end{itemize}
\end{sol}

\begin{prob}
    单个二能级原子 (上下能级分别为 $\lvert a\rangle$, $\lvert b\rangle$) 同单模光场 (频率 $\nu=\omega_{ab}$) 共振相互作用. 考虑偶极近似和旋转波近似, 假设相互作用系数为实数.
    \begin{itemize}
        \item[(1)] 写出半经典理论描述的原子-光场系统哈密顿量.
        \item[(2)] 写出全量子理论描述的原子-光场哈密顿量.
        \item[(3)] 原子初态为 $\lvert a\rangle$, 光场初态为真空态, 利用全量子理论的描述求 $t$ 时刻的原子布局数反转数 $W(t)=\abs{c_a}^2-\abs{c_b}^2$.
        \item[(4)] (附加题) 若原子初态为 $\frac{1}{\sqrt{2}}(\lvert a\rangle-\lvert b\rangle)$, 光场的初态为真空态, 求 $W(t)$.
    \end{itemize}
\end{prob}
\begin{sol}
    \begin{itemize}
        \item[(1)] 
        \item[(2)] 
        \item[(3)] 
        \item[(4)] 
    \end{itemize}
\end{sol}

\begin{prob}
    单模光场与热平衡辐射场热库相互作用, 其密度算符的运动方程为:
    \[
        \rho=-\frac{\gamma}{2}\bar{n}(aa^{\dagger}\rho-2a^{\dagger}\rho a+\rho aa^{\dagger})-\frac{\gamma}{2}(\bar{n}+1)(a^{\dagger}a\rho-2a\rho a^{\dagger}+\rho a^{\dagger}a).
    \]
    求 $t$ 时刻光场的粒子数平均值 $\langle a^{\dagger}(t)a(t)\rangle$. [提示: $\frac{\mathrm{d}\langle a^{\dagger}a\rangle}{\mathrm{d}t}=\tr(\rho a^{\dagger}a)$]
\end{prob}
\begin{sol}
    
\end{sol}

\begin{prob}
    单模光场与热平衡辐射场热库相互作用的 Langevin 方程为
    \[
        \tilde{a}(t)=-\frac{\gamma}{2}\tilde{a}(t)+F_a(t),
    \]
    且已知: $\langle F_{\tilde{a}}^{\dagger}(t)\tilde{a}(t)\rangle_R+\langle\tilde{a}^{\dagger}(t)F_{\tilde{a}}(t)\rangle=\gamma\bar{n}$. 试求 $\langle\tilde{a}^{\dagger}(t)\tilde{a}(t)\rangle_R$.
\end{prob}
\begin{sol}
    
\end{sol}

\begin{prob}[附加题]
    如何理解原子谱线宽度和寿命的不确定性关系 $\Delta\omega\propto 1/\tau$.
\end{prob}
\begin{sol}
    原子上能级具有有限的寿命 $\tau$, 由不确定性原理, 其能量值具有不确定性 $\Delta E$, 满足
    \begin{align}
        \Delta E\cdot\tau\sim\frac{\hbar}{2},
    \end{align}
    故原子谱线, 即原子由上能级跃迁至下能级辐射的光子频率具有不确定性
    \begin{align}
        \Delta\omega=\frac{\Delta E}{\hbar}\propto\frac{1}{\tau}.
    \end{align}
\end{sol}

\begin{prob}[附加题]
    谈谈你对信息的理解.
\end{prob}
\begin{sol}
    信息即不确定度的减少, 信息的数量用香农熵
    \begin{align}
        H=-\sum_{k=1}^np_k\log_2p_k
    \end{align}
    来衡量.
\end{sol}

\begin{prob}[附加题]
    简述量子光学主要是研究什么问题? 在经典物理学中主要用哪一些学科研究这些内容 (例如可举普通物理和四大力学中的学科和工具).
\end{prob}
\begin{sol}
    量子光学主要研究光场的量子化和光与物质的相互作用.

    经典物理中有光学、原子物理学、电动力学、量子力学来研究这些内容.
\end{sol}
\end{document}