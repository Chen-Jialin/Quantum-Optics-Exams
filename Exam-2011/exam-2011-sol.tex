\documentclass{assignment}
\ProjectInfos{量子光学}{PHYS6251P}{2011 年}{期末考试}{}{陈稼霖}[https://github.com/Chen-Jialin]{SA21038052}

\begin{document}
\begin{prob}[20 分]
    产生湮灭算符 $a^{\dagger}$, $a$ 满足对易关系 $[a,a^{\dagger}]=1$, 且 $[a,(a^{\dagger})^n]=n(a^{\dagger})^{n-1}$, 试证:
    \begin{itemize}
        \item[(i)] $[a^{\dagger},a^m]=-ma^{m-1}$;
        \item[(ii)] $a(a^{\dagger})^na^ma^{\dagger}=(a^{\dagger})^{n+1}a^{m+1}+(m+n+1)(a^{\dagger})^na^m+mn(a^{\dagger})^{n-1}a^{m-1}$.
    \end{itemize}
\end{prob}
\begin{pf}
    \begin{itemize}
        \item[(i)] 利用数学归纳法证明:
        \begin{itemize}
            \item 当 $m=1$ 时,
            \begin{align}
                [a^{\dagger},a^1]=-1=-1\cdot a^{1-1}.
            \end{align}
            \item 假设当 $m=k$ 时,
            \begin{align}
                [a^{\dagger},a^k]=-ka^{k-1},
            \end{align}
            则当 $m=k+1$ 时,
            \begin{align}
                [a^{\dagger},a^{k+1}]=[a^{\dagger},a^k]a+a^k[a^{\dagger},a]=-ka^{k-1}a+a^k\cdot(-1)=-(k+1)a^k=-(k+1)a^{(k+1)-1}.
            \end{align}
        \end{itemize}
        综上, $[a^{\dagger},a^m]=-ma^{m-1}$.
        \item[(ii)] 
        \begin{align}
            \notag a(a^{\dagger})^na^ma^{\dagger}=&[(a^{\dagger})^na+n(a^{\dagger})^{n-1}]a^ma^{\dagger}=(a^{\dagger})^na^{m+1}a^{\dagger}+n(a^{\dagger})^{n-1}a^ma^{\dagger}\\
            \notag=&(a^{\dagger})^n[a^{\dagger}a^{m+1}+(m+1)a^m]+n(a^{\dagger})^{n-1}[a^{\dagger}a^m+ma^{m-1}]\\
            =&(a^{\dagger})^{n+1}a^{m+1}+(m+n+1)(a^{\dagger})^na^m+mn(a^{\dagger})^{n-1}a^{m-1}.
        \end{align}
    \end{itemize}
\end{pf}

\begin{prob}[20 分]
    有如下几种单模辐射场, 分别计算它们的光子数分布函数 $p(m)$:
    \begin{itemize}
        \item[(1)] 数态的叠加 $\lvert\psi\rangle=\frac{1}{\sqrt{2}}(\lvert 0\rangle-i\lvert 10\rangle)$;
        \item[(2)] $\rho=\sum_{n=0}^{\infty}\frac{e^{-\kappa}\kappa^n}{n!}\lvert n\rangle\langle n\rvert$, $\kappa\in\mathbb{R}^+$;
        \item[(3)] 湮灭掉一个光子的热光场 $\rho'=\frac{a\rho a^{\dagger}}{\tr[a\rho a^{\dagger}]}$, 其中 $\rho$ 是热光场, 即\\
        $\rho=\frac{1}{1+\langle n\rangle}\sum_{n=0}^{\infty}\left(\frac{\langle n\rangle}{1+\langle n\rangle}\right)^n\lvert n\rangle\langle n\rvert$, $\langle n\rangle$ 是 $\rho$ 光场的平均光子数.
    \end{itemize}
\end{prob}
\begin{sol}
    \begin{itemize}
        \item[(1)] 该叠加态的光子数分布函数为
        \begin{align}
            p(m)=\abs{\langle m\vert\psi\rangle}^2=\frac{1}{2}\abs{\delta_{m0}+\delta_{m,10}}^2.
        \end{align}
        \item[(2)] 该辐射场的光子数分布函数为
        \begin{align}
            p(m)=\langle m\rvert\rho\lvert m\rangle=\frac{e^{-\kappa}\kappa^m}{m!}.
        \end{align}
        \item[(3)] 
        \begin{align}
            a\rho a^{\dagger}=&\frac{1}{1+\langle n\rangle}\sum_{n=0}^{\infty}\left(\frac{\langle n\rangle}{1+\langle n\rangle}\right)^{n+1}(n+1)\lvert n\rangle\langle n\rvert,\\
            \notag\tr[a\rho a^{\dagger}]=&\frac{1}{1+\langle n\rangle}\sum_{n=0}^{\infty}\left(\frac{\langle n\rangle}{1+\langle n\rangle}\right)^{n+1}(n+1)=\frac{1}{1+\langle n\rangle}\sum_{n=1}^{\infty}x^nn,
        \end{align}
        其中 $x=\frac{\langle n\rangle}{1+\langle n\rangle}$, 令 $f=\sum_{n=1}^{\infty}x^nn$, 由于 $xf=\sum_{n=1}^{\infty}x^{n+1}n$, $f-xf=x+\sum_{n=2}^{\infty}x^n=x+\frac{x^2}{1-x}=\frac{x}{1-x}$, 故 $f=\frac{x}{(1-x)^2}$,
        \begin{align}
            \tr[a\rho a^{\dagger}]=&\langle n\rangle.
        \end{align}
        该湮灭掉一个光子的热光场的密度矩阵为
        \begin{align}
            \rho'=\frac{a\rho a^{\dagger}}{\tr[a\rho a^{\dagger}]}=\sum_{n=0}^{\infty}\frac{\langle n\rangle^n}{(1+\langle n\rangle)^{n+2}}(n+1)\lvert n\rangle\langle n\rvert.
        \end{align}
        其光子数分布函数为
        \begin{align}
            p(m)=\langle m\rvert\rho'\lvert m\rangle=\frac{\langle n\rangle^m}{(1+\langle n\rangle)^{m+2}}(m+1).
        \end{align}
    \end{itemize}
\end{sol}

\begin{prob}[20 分]
    试通过计算判断, 上题 (1) 中的辐射场的光子数分布为何种分布 (Poisson, Sub-Poisson, Super-Poisson)?
\end{prob}
\begin{sol}
    上题 (1) 中的辐射场二阶相关度为
    \begin{align}
        g^{(2)}(0)=\frac{\langle a^{\dagger}a^{\dagger}aa\rangle}{\langle a^{\dagger}a\rangle^2}=\frac{\frac{1}{\sqrt{2}}(\langle 0\rvert+i\langle 10\rvert)a^{\dagger}a^{\dagger}aa\frac{1}{\sqrt{2}}(\lvert 0\rangle-i\lvert 10\rangle)}{\left[\frac{1}{\sqrt{2}}(\langle 0\rvert+i\langle 10\rvert)a^{\dagger}a\frac{1}{\sqrt{2}}(\lvert 0\rangle-i\lvert 10\rangle)\right]^2}=\frac{9}{5}>1,
    \end{align}
    故该辐射场的光子数分布为超泊松分布.
\end{sol}

\begin{prob}[20 分]
    增加了一个光子的相干态 (Single-phtono-add coherent state (SPACS)) $\lvert\alpha,1\rangle=\frac{a^{\dagger}}{\sqrt{1+\abs{\alpha}^2}}\lvert\alpha\rangle$. 考虑该辐射场的两个厄米算符 $X_1=\frac{1}{2}(a+a^{\dagger})$, $X_2=\frac{1}{2i}(a-a^{\dagger})$. 它们分别对应于场的复振幅的实部和虚部. 证明:\\
    SPACS 态 $\lvert\alpha,1\rangle$ 当 $\abs{\alpha}>1$ 时是压缩态, (本题取 $\alpha\in\mathbb{R}^{\dagger}$).
\end{prob}
\begin{pf}
    
\end{pf}

\begin{prob}[20 分]
    考虑一个理想的光学腔, 腔里有单模辐射场 $\lvert\phi_0\rangle=\frac{1}{\sqrt{2}}(\lvert 0\rangle-i\lvert 1\rangle)$. 处于基态且与单模场共振的两能级原子 $\lvert\psi_i\rangle=\lvert b\rangle$ 进入该光学腔, 与辐射场发生反应, 反应过程中相互作用的哈密顿量为 $V=\hbar g(\sigma_+a+a^{\dagger}\sigma_-)$. 系统的演化方程为 $\Psi(t)=e^{-\frac{i}{\hbar}Vt}\lvert\psi_i\rangle\otimes\lvert\phi\rangle$. 反应一段时间后原子从腔中逸出. \textbf{经探测: 出射原子已经从腔中吸收一个光子而被激发, 且处于 $\lvert\psi_f\rangle=\lvert a\rangle$ 激发态.}
    \begin{itemize}
        \item[(1)] 计算该单模场初始时刻 $\lvert\phi_0\rangle$ 的平均光子数 $\bar{n}$;
        \item[(2)] 试讨论, 在腔中被吸收一个光子的情况下: 此时腔内的辐射场的平均光子数变为多少? 此时辐射场的光子数分布为何种分布 (Poisson, Sub-Poisson, Super-Poisson)?
    \end{itemize}
\end{prob}
\begin{sol}
    \begin{itemize}
        \item[(1)] 
        \item[(2)] 
    \end{itemize}
\end{sol}
\end{document}