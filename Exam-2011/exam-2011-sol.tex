\documentclass{assignment}
\ProjectInfos{量子光学}{PHYS6251P}{2011 年}{期末考试}{}{陈稼霖}[https://github.com/Chen-Jialin]{SA21038052}

\begin{document}
\begin{prob}[20 分]
    产生湮灭算符 $a^{\dagger}$, $a$ 满足对易关系 $[a,a^{\dagger}]=1$, 且 $[a,(a^{\dagger})^n]=n(a^{\dagger})^{n-1}$, 试证:
    \begin{itemize}
        \item[(i)] $[a^{\dagger},a^m]=-ma^{m-1}$;
        \item[(ii)] $a(a^{\dagger})^na^ma^{\dagger}=(a^{\dagger})^{n+1}a^{m+1}+(m+n+1)(a^{\dagger})^na^m+mn(a^{\dagger})^{n-1}a^{m-1}$.
    \end{itemize}
\end{prob}
\begin{pf}
    \begin{itemize}
        \item[(i)] 利用数学归纳法证明:
        \begin{itemize}
            \item 当 $m=1$ 时,
            \begin{align}
                [a^{\dagger},a^1]=-1=-1\cdot a^{1-1}.
            \end{align}
            \item 假设当 $m=k$ 时,
            \begin{align}
                [a^{\dagger},a^k]=-ka^{k-1},
            \end{align}
            则当 $m=k+1$ 时,
            \begin{align}
                [a^{\dagger},a^{k+1}]=[a^{\dagger},a^k]a+a^k[a^{\dagger},a]=-ka^{k-1}a+a^k\cdot(-1)=-(k+1)a^k=-(k+1)a^{(k+1)-1}.
            \end{align}
        \end{itemize}
        综上, $[a^{\dagger},a^m]=-ma^{m-1}$.
        \item[(ii)] 
        \begin{align}
            \notag a(a^{\dagger})^na^ma^{\dagger}=&[(a^{\dagger})^na+n(a^{\dagger})^{n-1}]a^ma^{\dagger}=(a^{\dagger})^na^{m+1}a^{\dagger}+n(a^{\dagger})^{n-1}a^ma^{\dagger}\\
            \notag=&(a^{\dagger})^n[a^{\dagger}a^{m+1}+(m+1)a^m]+n(a^{\dagger})^{n-1}[a^{\dagger}a^m+ma^{m-1}]\\
            =&(a^{\dagger})^{n+1}a^{m+1}+(m+n+1)(a^{\dagger})^na^m+mn(a^{\dagger})^{n-1}a^{m-1}.
        \end{align}
    \end{itemize}
\end{pf}

\begin{prob}[20 分]
    有如下几种单模辐射场, 分别计算它们的光子数分布函数 $p(m)$:
    \begin{itemize}
        \item[(1)] 数态的叠加 $\lvert\psi\rangle=\frac{1}{\sqrt{2}}(\lvert 0\rangle-i\lvert 10\rangle)$;
        \item[(2)] $\rho=\sum_{n=0}^{\infty}\frac{e^{-\kappa}\kappa^n}{n!}\lvert n\rangle\langle n\rvert$, $\kappa\in\mathbb{R}^+$;
        \item[(3)] 湮灭掉一个光子的热光场 $\rho'=\frac{a\rho a^{\dagger}}{\tr[a\rho a^{\dagger}]}$, 其中 $\rho$ 是热光场, 即\\
        $\rho=\frac{1}{1+\langle n\rangle}\sum_{n=0}^{\infty}\left(\frac{\langle n\rangle}{1+\langle n\rangle}\right)^n\lvert n\rangle\langle n\rvert$, $\langle n\rangle$ 是 $\rho$ 光场的平均光子数.
    \end{itemize}
\end{prob}
\begin{sol}
    \begin{itemize}
        \item[(1)] 该叠加态的光子数分布函数为
        \begin{align}
            p(m)=\abs{\langle m\vert\psi\rangle}^2=\frac{1}{2}\abs{\delta_{m0}+\delta_{m,10}}^2.
        \end{align}
        \item[(2)] 该辐射场的光子数分布函数为
        \begin{align}
            p(m)=\langle m\rvert\rho\lvert m\rangle=\frac{e^{-\kappa}\kappa^m}{m!}.
        \end{align}
        \item[(3)] 
        \begin{align}
            a\rho a^{\dagger}=&\frac{1}{1+\langle n\rangle}\sum_{n=0}^{\infty}\left(\frac{\langle n\rangle}{1+\langle n\rangle}\right)^{n+1}(n+1)\lvert n\rangle\langle n\rvert,\\
            \notag\tr[a\rho a^{\dagger}]=&\frac{1}{1+\langle n\rangle}\sum_{n=0}^{\infty}\left(\frac{\langle n\rangle}{1+\langle n\rangle}\right)^{n+1}(n+1)=\frac{1}{1+\langle n\rangle}\sum_{n=1}^{\infty}x^nn,
        \end{align}
        其中 $x=\frac{\langle n\rangle}{1+\langle n\rangle}$, 令 $f=\sum_{n=1}^{\infty}x^nn$, 由于 $xf=\sum_{n=1}^{\infty}x^{n+1}n$, $f-xf=x+\sum_{n=2}^{\infty}x^n=x+\frac{x^2}{1-x}=\frac{x}{1-x}$, 故 $f=\frac{x}{(1-x)^2}$,
        \begin{align}
            \tr[a\rho a^{\dagger}]=&\langle n\rangle.
        \end{align}
        该湮灭掉一个光子的热光场的密度矩阵为
        \begin{align}
            \rho'=\frac{a\rho a^{\dagger}}{\tr[a\rho a^{\dagger}]}=\sum_{n=0}^{\infty}\frac{\langle n\rangle^n}{(1+\langle n\rangle)^{n+2}}(n+1)\lvert n\rangle\langle n\rvert.
        \end{align}
        其光子数分布函数为
        \begin{align}
            p(m)=\langle m\rvert\rho'\lvert m\rangle=\frac{\langle n\rangle^m}{(1+\langle n\rangle)^{m+2}}(m+1).
        \end{align}
    \end{itemize}
\end{sol}

\begin{prob}[20 分]
    试通过计算判断, 上题 (1) 中的辐射场的光子数分布为何种分布 (Poisson, Sub-Poisson, Super-Poisson)?
\end{prob}
\begin{sol}
    上题 (1) 中的辐射场二阶相关度为
    \begin{align}
        g^{(2)}(0)=\frac{\langle a^{\dagger}a^{\dagger}aa\rangle}{\langle a^{\dagger}a\rangle^2}=\frac{\frac{1}{\sqrt{2}}(\langle 0\rvert+i\langle 10\rvert)a^{\dagger}a^{\dagger}aa\frac{1}{\sqrt{2}}(\lvert 0\rangle-i\lvert 10\rangle)}{\left[\frac{1}{\sqrt{2}}(\langle 0\rvert+i\langle 10\rvert)a^{\dagger}a\frac{1}{\sqrt{2}}(\lvert 0\rangle-i\lvert 10\rangle)\right]^2}=\frac{9}{5}>1,
    \end{align}
    故该辐射场的光子数分布为超泊松分布.
\end{sol}

\begin{prob}[20 分]
    增加了一个光子的相干态 (Single-phtono-add coherent state (SPACS)) $\lvert\alpha,1\rangle=\frac{a^{\dagger}}{\sqrt{1+\abs{\alpha}^2}}\lvert\alpha\rangle$. 考虑该辐射场的两个厄米算符 $X_1=\frac{1}{2}(a+a^{\dagger})$, $X_2=\frac{1}{2i}(a-a^{\dagger})$. 它们分别对应于场的复振幅的实部和虚部. 证明:\\
    SPACS 态 $\lvert\alpha,1\rangle$ 当 $\abs{\alpha}>1$ 时是压缩态, (本题取 $\alpha\in\mathbb{R}^+$).
\end{prob}
\begin{pf}
    $X_1$ 的均值:
    \begin{align}
        \notag\langle X_1\rangle=&\langle\alpha\rvert\frac{a}{\sqrt{1+\abs{\alpha}^2}}\frac{1}{2}(a+a^{\dagger})\frac{a^{\dagger}}{\sqrt{1+\abs{\alpha}^2}}\lvert\alpha\rangle=\frac{1}{2(1+\alpha^2)}\langle\alpha\rvert(aaa^{\dagger}+aa^{\dagger}a^{\dagger})\lvert\alpha\rangle\\
        \notag=&\frac{1}{2(1+\alpha^2)}\langle\alpha\rvert[a(a^{\dagger}a+1)+(a^{\dagger}a+1)a^{\dagger}]\lvert\alpha\rangle=\frac{1}{2(1+\alpha^2)}\langle\alpha\rvert(aa^{\dagger}a+a+a^{\dagger}aa^{\dagger}+a^{\dagger})\lvert\alpha\rangle\\
        \notag=&\frac{1}{2(1+\alpha^2)}\langle\alpha\rvert[(a^{\dagger}a+1)a+a+a^{\dagger}(a^{\dagger}a+1)+a^{\dagger}]\lvert\alpha\rangle=\frac{1}{2(1+\alpha^2)}\langle\alpha\rvert(a^{\dagger}aa+2a+a^{\dagger}a^{\dagger}a+2a^{\dagger})\lvert\alpha\rangle\\
        =&\frac{\alpha(2+\alpha^2)}{1+\alpha^2}.
    \end{align}
    $X_1^2$ 的均值:
    \begin{align}
        \notag\langle X_1^2\rangle=&\langle\alpha\rvert\frac{a}{\sqrt{1+\abs{\alpha}^2}}\left[\frac{1}{2}(a+a^{\dagger})\right]^2\frac{a^{\dagger}}{\sqrt{1+\abs{\alpha}^2}}\lvert\alpha\rangle=\frac{1}{4(1+\alpha^2)}\langle\alpha\rvert(aaaa+aaa^{\dagger}a^{\dagger}+aa^{\dagger}aa^{\dagger}+aa^{\dagger}a^{\dagger}a^{\dagger})\lvert\alpha\rangle\\
        \notag=&\frac{1}{4(1+\alpha^2)}\langle\alpha\rvert[aaaa+a(a^{\dagger}a+1)a^{\dagger}+aa^{\dagger}aa^{\dagger}+(a^{\dagger}a+1)a^{\dagger}a^{\dagger}]\lvert\alpha\rangle\\
        \notag=&\frac{1}{4(1+\alpha^2)}\langle\alpha\rvert(aaaa+2aa^{\dagger}aa^{\dagger}+aa^{\dagger}+a^{\dagger}aa^{\dagger}a^{\dagger}+a^{\dagger}a^{\dagger})\lvert\alpha\rangle\\
        \notag=&\frac{1}{4(1+\alpha^2)}\langle\alpha\rvert(aaaa+2(a^{\dagger}a+1)(a^{\dagger}a+1)+(a^{\dagger}a+1)+a^{\dagger}(a^{\dagger}a+1)a^{\dagger}+a^{\dagger}a^{\dagger})\lvert\alpha\rangle\\
        \notag=&\frac{1}{4(1+\alpha^2)}\langle\alpha\rvert(aaaa+2a^{\dagger}aa^{\dagger}a+5a^{\dagger}a+3+a^{\dagger}a^{\dagger}aa^{\dagger}+2a^{\dagger}a^{\dagger})\lvert\alpha\rangle\\
        \notag=&\frac{1}{4(1+\alpha^2)}\langle\alpha\rvert(aaaa+2a^{\dagger}(a^{\dagger}a+1)a+5a^{\dagger}a+3+a^{\dagger}a^{\dagger}(a^{\dagger}a+1)+2a^{\dagger}a^{\dagger})\lvert\alpha\rangle\\
        \notag=&\frac{1}{4(1+\alpha^2)}\langle\alpha\rvert(aaaa+2a^{\dagger}a^{\dagger}aa+7a^{\dagger}a+3+a^{\dagger}a^{\dagger}a^{\dagger}a+3a^{\dagger}a^{\dagger})\lvert\alpha\rangle\\
        =&\frac{4\alpha^4+10\alpha^2+3}{4(1+\alpha^2)}.
    \end{align}
    $X_1$ 的涨落:
    \begin{align}
        \Delta X_1=\sqrt{\langle X_1^2\rangle-\langle X_1\rangle^2}=\frac{\sqrt{-2\alpha^4-3\alpha^2+3}}{2(1+\alpha^2)}.
    \end{align}
    若 $\lvert\alpha,1\rangle$ 为压缩态, 则
    \begin{gather}
        \Delta X_1\neq\frac{1}{2},\\
        \Longrightarrow\alpha\neq 1.
    \end{gather}
    故当 $\abs{\alpha}>1$ 时, $\lvert\alpha,1\rangle$ 为压缩态.
\end{pf}

\begin{prob}[20 分]
    考虑一个理想的光学腔, 腔里有单模辐射场 $\lvert\phi_0\rangle=\frac{1}{\sqrt{2}}(\lvert 0\rangle-i\lvert 1\rangle)$. 处于基态且与单模场共振的两能级原子 $\lvert\psi_i\rangle=\lvert b\rangle$ 进入该光学腔, 与辐射场发生反应, 反应过程中相互作用的哈密顿量为 $\mathscr{V}=\hbar g(\sigma_+a+a^{\dagger}\sigma_-)$. 系统的演化方程为 $\Psi(t){A+F}=e^{-\frac{i}{\hbar}\mathscr{V}t}\lvert\psi_i\rangle\otimes\lvert\phi\rangle$. 反应一段时间后原子从腔中逸出. \textbf{经探测: 出射原子已经从腔中吸收一个光子而被激发, 且处于 $\lvert\psi_f\rangle=\lvert a\rangle$ 激发态.}
    \begin{itemize}
        \item[(1)] 计算该单模场初始时刻 $\lvert\phi_0\rangle$ 的平均光子数 $\bar{n}$;
        \item[(2)] 试讨论, 在腔中被吸收一个光子的情况下: 此时腔内的辐射场的平均光子数变为多少? 此时辐射场的光子数分布为何种分布 (Poisson, Sub-Poisson, Super-Poisson)?
    \end{itemize}
\end{prob}
\begin{sol}
    \begin{itemize}
        \item[(1)] 该单模场初始时刻 $\lvert\phi_0\rangle$ 的平均光子数为
        \begin{align}
            \bar{n}=\langle\phi_0\rvert a^{\dagger}a\lvert\phi_0\rangle=\frac{1}{2}.
        \end{align}
        \item[(2)] 无微扰哈密顿量为
        \begin{align}
            \hat{H}_0=\hbar\nu a^{\dagger}a+\frac{1}{2}\hbar\omega\sigma_z.
        \end{align}
        相互作用绘景中, 相互作用哈密顿量为
        \begin{align}
            \hat{V}=e^{i\hat{H}_0t/\hbar}\mathscr{V}e^{-i\hat{H}_0t/\hbar}=\hbar g(\sigma_+ae^{i\Delta t}+a^{\dagger}\sigma_-e^{-i\Delta t}),
        \end{align}
        其中 $\Delta=\omega-\nu$.
        将系统的量子态
        \begin{align}
            \lvert\Psi(t)\rangle_{A+F}=\sum_{n=0}^{\infty}(C_{a,n}(t)\lvert a,n\rangle+C_{b,n}(t)\lvert b,n\rangle)
        \end{align}
        代入薛定谔方程
        \begin{align}
            i\hbar\frac{\partial}{\partial t}\lvert\Psi(t)\rangle=\hat{V}\lvert\psi(t)\rangle
        \end{align}
        有
        \begin{align}
            \dot{C}_{a,n}(t)=&-igC_{b,n+1}\sqrt{n+1}e^{i\Delta t},\\
            \dot{C}_{b,n}(t)=&-igC_{a,n}\sqrt{n+1}e^{-i\Delta t},
        \end{align}
        解得
        \begin{align}
            C_{a,n}(t)=&\left\{C_{a,n}(0)\left[\cos\left(\frac{\Omega_nt}{2}\right)-\frac{i\Delta}{\Omega_n}\sin\left(\frac{\Omega_nt}{2}\right)\right]-\frac{2ig\sqrt{n+1}}{\Omega_n}C_{b,n+1}(0)\sin\left(\frac{\Omega_nt}{2}\right)\right\}e^{i\Delta t/2},\\
            C_{b,n+1}=&\left\{C_{b,n+1}(0)\left[\cos\left(\frac{\Omega_nt}{2}\right)+\frac{i\Delta}{\Omega_n}\sin\left(\frac{\Omega_nt}{2}\right)\right]-\frac{2ig\sqrt{n+1}}{\Omega_n}C_{a,n}(0)\sin\left(\frac{\Omega_nt}{2}\right)\right\}e^{-i\Delta t/2},
        \end{align}
        其中 $\Omega_n^2=\Delta^2+4g^2(n+1)$.
        考虑到系统的初始状态
        \begin{align}
            \lvert\Psi(0)\rangle_{A+F}=\lvert b\rangle\otimes\frac{1}{\sqrt{2}}(\lvert 0\rangle-i\lvert 1\rangle),
        \end{align}
        即 $C_{b,0}(0)=\frac{1}{\sqrt{2}}$, $C_{b,1}=-\frac{i}{\sqrt{2}}$,
        故
        \begin{align}
            C_{a,0}(t)=&-\frac{\sqrt{2}g\sqrt{n+1}}{\Omega_n}\sin\left(\frac{\Omega_nt}{2}\right)e^{i\Delta t/2},\\
            C_{b,1}(t)=&-\frac{i}{\sqrt{2}}\left[\cos\left(\frac{\Omega_nt}{2}\right)+\frac{i\Delta}{\Omega_n}\sin\left(\frac{\Omega_nt}{2}\right)\right]e^{-i\Delta t/2},
        \end{align}
        即系统的量子态演化方程为
        \begin{align}
            \notag\lvert\Psi(t)\rangle_{A+F}=&-\frac{\sqrt{2}g\sqrt{n+1}}{\Omega_n}\sin\left(\frac{\Omega_nt}{2}\right)e^{i\Delta t/2}\lvert a,0\rangle-\frac{i}{\sqrt{2}}\left[\cos\left(\frac{\Omega_nt}{2}\right)+\frac{i\Delta}{\Omega_n}\sin\left(\frac{\Omega_nt}{2}\right)\right]e^{-i\Delta t/2}\lvert b,1\rangle\\
            &+\frac{1}{\sqrt{2}}\lvert b,0\rangle.
        \end{align}
        当探测得 $\lvert\psi_f\rangle=\lvert a\rangle$ 时, 系统的量子态塌缩至
        \begin{align}
            \lvert\Psi_f\rangle=\lvert a,0\rangle.
        \end{align}
        此时腔内的辐射场的平均光子数为 $0$, 辐射场的光子数分布为亚泊松分布.
    \end{itemize}
\end{sol}
\end{document}